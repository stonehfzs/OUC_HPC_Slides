\documentclass[aspectratio=169]{beamer}
\usepackage{booktabs}
\usepackage{listings}
\usepackage{graphicx}
\usepackage{float}
\usepackage{subfigure}
\usepackage{fontspec}

%\usetheme[progressbar=frametitle]{Madrid}
\usetheme{metropolis}
\definecolor{whalebluelight}{RGB}{0,191,255}
\definecolor{whalebluenormal}{RGB}{113,221,255}
\setbeamercolor{palette primary}{bg=whalebluenormal}
\setbeamercolor{normal text}{fg=black}
\usecolortheme[named=whalebluelight]{structure}
\setsansfont{Latin Modern Sans}
\setmonofont{Latin Modern Mono}
\linespread{1.2}

\lstset{
    basicstyle = \ttfamily,
    columns = fullflexible,
}

\title{R Introduction}
\subtitle{Basic R Introductions and Operations}
\date{\today}
\author{Jingyu Sun}
\institute{\small \emph{Ocean University of China, High Performance Computing Club, Qingdao 266100}}

\begin{document}
    \begin{frame}
        \titlepage
    \end{frame}

    \begin{frame}{What is R?}
        Initially R language was originied from S language, which was developed at Bell Laboratories by John Chambers and colleagues.\par
        R language is designed for data analysis and statistical computing, which is a free software environment for statistical computing and graphics. Now it is a part of GNU project (so that's why I like it).\par
        R language is widely used among statisticians and data miners for developing statistical software and data analysis.\par
        Other hot statistical languages: Python (we will discuss later), Julia (especially for high performance computing), Stata (designed for applcations), etc.\par
    \end{frame}

    \begin{frame}{R Installation}
        Steps for install R and RStudio:
        \begin{itemize}
            \item Download R from CRAN (TUNA mirror): \url{https://mirrors.tuna.tsinghua.edu.cn/CRAN/bin/windows/base/R-4.4.3-win.exe}
            \item Install R
            \item Download RStudio (A Integrated Developer Environment for R) \url{https://posit.co/download/rstudio-desktop/} (Remember to select the version that suit your systemm)
            \item Install RStudio
        \end{itemize}
    \end{frame}

    \begin{frame}[fragile]{Begin with R: read and write data}
        Basically everytime we use R is for data analysing, so we need to know how to read and write data.\par
        Suppose we have a data file named \texttt{iris.csv}, we can use the following code to read the data into R:\par
        \begin{lstlisting}[language=R]
        iris <- read.csv("iris.csv")
        \end{lstlisting}
    \end{frame}

    \begin{frame}[fragile]{Begin with R: read and write data}
        Now wondering if we have the iris data and we want to write it into a file, we can use the following code:\par
        \begin{lstlisting}[language=R]
        write.csv(iris, "iris2.csv")
        \end{lstlisting}
    \end{frame}

    \begin{frame}{Begin with R: read and write data}
        Now as you can see, read and write are probably easier than any other languages, which means that R is a good language for data analysis.\par
        To learn this further, let's move on how to do basic operations in R.\par
    \end{frame}

    \begin{frame}{Basic Operations: Structure of data}
        The most common scene we would face is we do not know how the data is strcutured. So we need to know how to see the structure of data.\par
        \begin{itemize}
            \item \texttt{head(iris)}: Show the first 6 rows of the data
            \item \texttt{tail(iris)}: Show the last 6 rows of the data
            \item \texttt{str(iris)}: Show the structure of the data
            \item \texttt{summary(iris)}: Show the summary of the data
        \end{itemize}
    \end{frame}

    \begin{frame}{Basic Operations: Access Data}
        Also, similar to other languages, R has some methods to access data in the data frame.\par
        \begin{itemize}
            \item \texttt{iris\$SepalLengthCm}: Access the column named SepalLengthCm
            \item \texttt{iris[1, ]}: Access the first row of the data
            \item \texttt{iris[1:5, ]}: Access the first 5 rows of the data
            \item \texttt{iris[iris\$SepalLengthCm > 6, ]}: Access the rows that SepalLengthCm is greater than 6
        \end{itemize}
    \end{frame}

    \begin{frame}{Basic Operations: Statistics of Data}
        While we already knows how to access the data, we can do some statistics on the data.\par
        \begin{itemize}
            \item \texttt{mean(iris\$SepalLengthCm)}: Calculate the mean of SepalLengthCm
            \item \texttt{sd(iris\$SepalLengthCm)}: Calculate the standard deviation of SepalLengthCm
            \item \texttt{cor(iris\$SepalLengthCm, iris\$SepalWidthCm)}: Calculate the correlation between SepalLengthCm and SepalWidthCm
            \item \texttt{table(iris\$Species)}: Calculate the frequency of each species
        \end{itemize}
    \end{frame}

    \begin{frame}{Basic Operations: Matrices and Arraies}
        R also supports matrices and arraies, which are very useful in data analysis.\par
        \begin{itemize}
            \item \texttt{matrix(1:9, nrow=3, ncol=3)}: Create a 3x3 matrix
            \item \texttt{array(1:9, dim=c(3, 3, 3))}: Create a 3x3x3 array
            \item \texttt{t(matrix(1:9, nrow=3, ncol=3))}: Transpose the matrix
            \item \texttt{solve(matrix(1:4, nrow=2, ncol=2))}: Inverse the matrix
        \end{itemize}
    \end{frame}

    \begin{frame}{Basic Operations: Assign Values}
        The previous code are all about how to get the data, but how to assign values to the data?\par
        \begin{itemize}
            \item \texttt{iris\$SepalLengthCm[1] <- 5.1}: Assign 5.1 to the first row of SepalLengthCm
            \item \texttt{iris[1, 1] <- 5.1}: Assign 5.1 to the first row of the first column
            \item \texttt{iris[iris\$SepalLengthCm > 6, 1] <- 6}: Assign 6 to the rows that SepalLengthCm is greater than 6
        \end{itemize}
    \end{frame}

    \begin{frame}{Basic Operations: Assign Values}
        Another operator related to this is \texttt{\%<\%}, which is used to assign values to a variable.\par
        This operator means that the value on the right side is assigned to the variable on the left side.\par
        \begin{itemize}
            \item \texttt{a \%<\% 5}: Assign 5 to a
            \item \texttt{a \%<\% b \%<\% 5}: Assign 5 to b and a
        \end{itemize}
    \end{frame}

    \begin{frame}{Work Directory}
        Every R session has a working directory, which is the default directory for file input and output.\par
        We will simply use the configuration in R Studio to set the working directory.\par
    \end{frame}

    \begin{frame}{Progamming Logic}
        It's not often use the programming logic in R, but it's still useful to know how to use it.\par
        \begin{itemize}
            \item \texttt{if (condition) \{ ... \} else \{ ... \}}: If the condition is true, execute the first block, otherwise execute the second block
            \item \texttt{for (i in 1:10) \{ ... \}}: Execute the block 10 times
            \item \texttt{while (condition) \{ ... \}}: Execute the block until the condition is false
            \item \texttt{break}: Break the loop
        \end{itemize}
    \end{frame}

    \begin{frame}{Applications of R: Machine Learning (Linear Regression)}
        The basic machine learning method is regression, which here we will discuss linear regression. Still, we will use iris data as an example.\par
        \begin{itemize}
            \item \texttt{lm <- lm(SepalLengthCm \textasciitilde SepalWidthCm, data=iris)}: Fit a linear model to the data
            \item \texttt{predict(lm, newdata=iris)}: Predict
            \item \texttt{summary(lm)}: Show the summary
            \item \texttt{plot(lm)}: Plot 
            \item \texttt{abline(lm)}
            \item \texttt{anova(lm)}: Show the ANOVA
            \item \texttt{confint(lm)}: Show the CI
        \end{itemize}
    \end{frame}


    \begin{frame}{Applications of R: Machine Learning (Nonlinear Regression)}
        Also we could conduct some nonlinear regression.\par
        \begin{itemize}
            \item \texttt{lm <- lm(SepalLengthCm \textasciitilde{} SepalWidthCm\textasciicircum{}2, data=iris)}: Fit a linear model to the data
            \item \texttt{predict(lm, newdata=iris)}: Predict
            \item \texttt{summary(lm)}: Show the summary
            \item \texttt{plot(lm)}: Plot 
            \item \texttt{abline(lm)}
            \item \texttt{anova(lm)}: Show the ANOVA
            \item \texttt{confint(lm)}: Show the CI
        \end{itemize}
    \end{frame}

    \begin{frame}{Applications of R: Machine Learning (Random Forest)}
        Other methods like random forest are also available in R.\par
        \begin{itemize}
            \item \texttt{library(randomForest)}: Load the random forest library
            \item \texttt{rf <- randomForest(SepalLengthCm \textasciitilde SepalWidthCm + PetalLengthCm + PetalWidthCm, data=iris)}: Fit a random forest model to the data
            \item \texttt{predict(rf, newdata=iris)}: Predict
            \item \texttt{importance(rf)}: Variable importance
            \item \texttt{plot(rf)}: Plot 
        \end{itemize}
    \end{frame}

    \begin{frame}{Summary of R Language}
        Now we have discussed some basic operations in R, and also some applications of R. Later we would like to introduce some further applications by R.\par
        The reason that why R does not take much time is that R is a very simple language (for learners), and it is initially designed for mathematicians, not for computing.\par
    \end{frame}
    
    \begin{frame}
        \begin{center}
            This work by Ocean University of China, High Performance Computing Club is licensed under CC BY-NC 4.0. To view a copy of this license, visit \url{https://creativecommons.org/licenses/by-nc/4.0/}.\par
            \includegraphics{by-nc.png}
        \end{center}
    \end{frame}
\end{document}